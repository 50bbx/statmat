\documentclass[12pt]{article}

\title{Statistica Matematica}
\author{Git}
\date{}		

\usepackage{amsmath}%

\newtheorem{corollary}{Corollary}
\newtheorem{definition}{Definition}
\newtheorem{example}{Example}
\newtheorem{exercise}{Exercise}
\newtheorem{problem}{Problem}
\newtheorem{proposition}{Proposition}
\newtheorem{remark}{Remark}
\newtheorem{summary}{Summary}
\newtheorem{theorem}{Theorem}

\begin{document}
\maketitle


\section{Introduzione}
\subsection{Funzione generatrice dei momenti}
\subsection{Famiglia esponenziale a k parametri}
\subsection{Trasformazioni di variabili casuali}

\begin{theorem}
Sia X una vc con funzione di massa $f_X(x)=P(X=x)$ e sia W=h(X) una nuova vc, allora $P(W=w)=\sum_{{x:h(x)=w}}P(X=x)$ per ogni punto del supporto di X.
\end{theorem}
\begin{example}
Sia X $\sim$ b(n,p), e quindi $f_X(x,p)=\left(\stackrel{n}{p}\right)p^x(1-p)^{n-x}$
\end{example}

\end{document}