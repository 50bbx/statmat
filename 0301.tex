%%%%lezione 1 marzo%%%%

\subsection{Trasformazioni di variabili casuali}
Lezione del 01/03, ultima modifica 09/04, Michele Nardin
\paragraph{Discrete}
\begin{teo}
Sia X una vc con funzione di massa $f_X(x)=P(X=x)$, e sia $A_X$ il suo supporto. 
Sia W=h(X) una nuova vc. Allora $$P(W=w)=\sum_{\{x\in A_X:h(x)=w\}}P(X=x)$$
\end{teo}

\noindent \textbf{Esempi}
\begin{enumerate}

\item Sia $X \sim b(n,p)$ con relativa funzione di massa 
$f_X(x,p)=\binom{n}{x}p^x(1-p)^{n-x}\mathbbm{1}_{0,1,...,n}(x)$,
$n$ noto e $p\in(0,1)$.

Considero quindi $W=n-X$. Come si distribuisce $W$? 
$$P(W=w)=P(X=n-w)=\binom{n}{n-w}p^{n-w}(1-p)^w\mathbbm{1}_{0,1,...,n}(w)$$
\item Sia $X$ una vc tale che 
$f_X(x)=P(X=x)=\left(\frac{1}{2}\right)^x\mathbbm{1}_\mathbbm{N}(x)$, $W=X^3$. 
%%lasciando una riga vuota va a capo senza dare badbox! figata
$$P(W=w)=P(X^3=w)=P(X=\sqrt[3]{w})=\left(\frac{1}{2}\right)^{\sqrt[3]{w}}\mathbbm{1}_{1,8,27,64,...}(w)$$
\end{enumerate}

\paragraph{Assolutamente continue}
\begin{teo}
Sia X una variabile casuale (ass continua) con funzione di densità $f_X(x)$ e sia $W=h(X)$, ove $h$ è una funzione monotona. 
Supponiamo inoltre che $f_X(x)$ sia continua sul supporto di X e che $h^{-1}(w)$ abbia derivata continua sul supporto di W. Allora
$$f_W(w)=f_X(h^{-1}(w))\left|\frac{d}{dw}h^{-1}(w)\right|\mathbbm{1}_{A_W}(w)$$
\end{teo}
\noindent \textbf{Esempio}
(Standardizzazione di una vc normale)
Sia $X \sim N(m,s^2)$. 
Considero $W=h(X)=\frac{X-m}{s}$. 
Allora, dato che $h^{-1}(w)=sw+m$, 
che ha derivata continua su tutto $\mathbbm{R}$,  
$$f_W(w)=f_X(sw+m)|s|=\frac{e^{\frac{-w^2 }{2}}}{\sqrt{2\pi}} = f_{N(0,1)}$$
\begin{teo}
Se $W=h(X)$ ove h è monotona a tratti (un numero di tratti finito k) e valgono le condizioni del teorema precedente (su ogni tratto), allora
$$f_W(w)=\sum_{n=1}^k f_X(h_n^{-1}(w))\left|\frac{d}{dw}h_n^{-1}(w)\right|\mathbbm{1}_{A_W}(w)$$
\end{teo}
\noindent \textbf{Esempio} (Chi-quadro)

\noindent Sia $X \sim N(0,1)$ e $W=h(X)=X^2$. $h$ è monotona sui tratti $A_0={0}$, $A_1=(-\infty,0)$, $A_2=(0,+\infty)$. 

\noindent Considero $h_1(x)=x^2$ per $x<0$ mentre $h_2(x)=x^2$ per $x>0$.

\noindent Trovo che $h_1^{-1}(w)=-\sqrt{w}$ (NB:$h_1^{-1}(w)\in A_1 \forall w \geq 0$), mentre $h_2^{-1}(w)=\sqrt{w}$ (NB:$h_2^{-1}(w)\in A_2 \forall w \geq 0$).
 
\noindent $\frac{d}{dw} h_1^{-1}(w)=-\frac{1}{2\sqrt{w}}$, $\frac{d}{dw} h_2^{-1}(w)=\frac{1}{2\sqrt{w}}$ sono entrambe continue su $\mathbbm{R}_+$.
$$f_W(w)=\frac{1}{\sqrt{2\pi}}e^{\frac{-(-\sqrt{w})^2 }{2}}\left|\frac{1}{2\sqrt{w}}\right|+\frac{1}{\sqrt{2\pi}}e^{\frac{-(\sqrt{w})^2 }{2}}\left|\frac{1}{2\sqrt{w}}\right|$$ 
$$=\frac{1}{\sqrt{2\pi w}}
e^{\frac{-w}{2}}\mathbbm{1}_{\mathbbm{R}+}
(w)=\frac{1}{2^{1/2}\Gamma(1/2)}
w^{\frac{1}{2}-1}e^{\frac{-w}{2}}$$
Si riconosce che $W \sim \mathcal{G}(\alpha=1/2,\beta=2)$ e si chiama Chi quadrato con $\nu=1$ gradi di libertà.

In generale, una vc Chi Quadro con $\nu=n$ gradi di libertà è $W=\sum_{i=1}^n X_i^2$, ove $X_1,X_2,...,X_n$ sono vc iid N(0,1). Per il Teorema 2 sulla FGM di una somma di vc iid si trova immediatamente che $W \sim \mathcal{G}(\alpha=n \cdot 1/2,\beta=2)$.
\subsection{Convergenze}
\paragraph{Convergenza in probabilità}
\begin{definizione}

Sia $\{X_n\}_{n\in\mathbbm{N}}$ una successione di variabili casuali e 
sia X un'altra variabile casuale, tutte definite sullo stesso spazio campionario.
Diciamo che $X_n$ converge in probabilità a X (scriviamo $X_n\stackrel{p}{\rightarrow}X$) se $\forall\varepsilon>0$ $$\lim_{n \rightarrow\infty} P(|X_n-X|\geq\varepsilon)=0$$
\end{definizione}
\begin{oss}
Se $X_n\stackrel{p}{\rightarrow}X$ diciamo che la "massa" della differenza $|X_n-X|$ converge a 0.
Inoltre, quando scriviamo $X_n\stackrel{p}{\rightarrow}X$, stiamo sottintendendo tutta la parte iniziale della definizione precendete, cioè il "sia $\{X_n\}_{n\in\mathbbm{N}}$ una successione di variabili casuali...".
\end{oss}
\begin{teo} Alcuni risultati utili: 

\begin{enumerate}
\item Supponiamo che  $X_n\stackrel{p}{\rightarrow}X$ e  $Y_n\stackrel{p}{\rightarrow}Y$. Allora  $X_n+Y_n\stackrel{p}{\rightarrow}X+Y$
\item Supponiamo che $X_n\stackrel{p}{\rightarrow}X$ e sia a una costante. Allora $aX_n\stackrel{p}{\rightarrow}aX$ 
\item Supponiamo che  $X_n\stackrel{p}{\rightarrow}a$ costante, e sia g una funzione reale continua in a. Allora  $g(X_n)\stackrel{p}{\rightarrow}g(a)$
\item (Corollario di 3.) Se $X_n\stackrel{p}{\rightarrow}a$, allora $X_n^2\stackrel{p}{\rightarrow}a^2$, $\frac{1}{X_n}\stackrel{p}{\rightarrow}\frac{1}{a}$ (se $a\neq0$), $\sqrt{X_n}\stackrel{p}{\rightarrow}\sqrt{a}$ ($a\geq0$).
\item $X_n\stackrel{p}{\rightarrow}X$ e $Y_n\stackrel{p}{\rightarrow}Y$ allora $X_nY_n\stackrel{p}{\rightarrow}XY$
\end{enumerate}
\end{teo}
%\textbf{Esempi}
%\begin{enumerate}
%\item
%\item
%\item
%\end{enumerate}
\paragraph{Convergenza in distribuzione}
\begin{definizione}

Sia $\{X_n\}_{n\in\mathbbm{N}}$ una successione di variabili casuali e 
sia X un'altra variabile casuale, tutte definite sullo stesso spazio campionario.

Siano $F_{X_n}$ e $F_X$ le relative funzioni di ripartizione (dette anche ''di distribuzione'').
Sia $C(F_X)$ l'insieme dei punti ove $F_X$ è continua. 
Diciamo che $X_n$ converge in distribuzione (o in legge) a X (scriviamo $X_n\stackrel{d}{\rightarrow}X$) se 
$$\lim_{n \rightarrow\infty} F_{X_n}(x)=F_X(x) \; \; \; \forall x \in C(F_X)$$
\end{definizione}

\begin{teo}
Se $X_n\stackrel{p}{\rightarrow}X$ allora $X_n\stackrel{d}{\rightarrow}X$.
\end{teo}
\begin{oss}
Il contrario in generale non vale, tranne nel caso in cui X è una vc degenere (cioè costante).
\end{oss}
\begin{teo} Supponiamo che $X_n\stackrel{d}{\rightarrow}X$ e sia g una funzione continua sul supporto di X. Allora $g(X_n)\stackrel{d}{\rightarrow}g(X)$
\end{teo}
\begin{teo} [Slutsky] Supponiamo che $X_n\stackrel{d}{\rightarrow}X$, $A_n\stackrel{p}{\rightarrow}a$ costante e $B_n\stackrel{p}{\rightarrow}b$ costante. Allora $A_n+B_n X_n\stackrel{d}{\rightarrow}a+bX$
\end{teo}
%%%%fine lezione del 1 marzo%%%%