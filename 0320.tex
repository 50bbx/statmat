%%%lezione 20 marzo
\\ \\
Lezione del 20 marzo, ultima modifica 26 marzo, Michele Nardin

\subsubsection{Intervalli di confidenza per rapporti di varianze}

\textbf{Distribuzione F di Snedecor-Fisher}
\\ \\
Siano $W_1 \sim \chi^2_{\nu_1}$ e $W_2 \sim \chi^2_{\nu_2}$ indipendenti.
La distribuzione $F_{\nu_1,\nu_2}$ è definita come il rapporto tra due chiquadrato divise per i
rispettivi gradi di libertà, in formule $$W=\frac{W_1/\nu_{1}}{W_2/\nu_{2}}$$ ed ha funzione di densità $$f_W(w;\nu_1,\nu_2)=\frac{\Gamma(\frac{\nu_1 + \nu_2}{2})}{\Gamma(\frac{\nu_1}{2})+\Gamma(\frac{\nu_2}{2})} (\nu_1 / \nu_2)^{\nu_1 / 2} w^{\nu_1 / 2 - 1} \left[ 1 - \frac{\nu_1}{\nu_2} w  \right]^{\frac{\nu_1 + \nu_2}{2}} \mathbbm{1}_{\mathbbm{R}^+} (w)$$
Vale la seguente proprietà, utile per calcolare i quantili non tabulati:
$$Se \; W \sim F_{\nu_1,\nu_2} \Rightarrow \frac{1}{W} \sim F_{\nu_2,\nu_1}$$
Le tavole (comunemente) forniscono i valori dei quantili per $(1-\alpha) \in {0.80,0.90,0.95,0.975,0.99,0.999}$. Quindi possiamo sfruttare la proprietà sopra scritta per trovare che $$w_{\alpha;\nu_1,\nu_2}=\frac{1}{w_{1-\alpha;\nu_1,\nu_2}}$$
\\ \\
\textbf{Intervallo di confidenza per rapporti di varianze}
\\
 Supponiamo di avere $(X_1,...,X_{n_1})$ da distribuzione normale con media $\mu_1$ e varianza $\sigma_1^2$ (ignote), i loro stimatori $\bar{X}$ e $S^2_1$ e
 $(Y_1,...,Y_{n_2})$ da distribuzione normale con media $\mu_2$ e varianza $\sigma_2^2$ (ignote) e i loro stimatori $\bar{Y}$ e $S^2_2$.
 Ricordiamo che $\frac{(n_1-1)S^2_1}{\sigma^2_1} \sim \chi^2_{n_1 - 1}$ (idem per $S^2_2)$.
 
 Fissato $1-\alpha$ consideriamo una statistica pivot e $w_1$, $w_2$ tc $P(w_1 \leq W \leq w_2)=1-\alpha$.
 La statistica pivot in questione sarà $$W=\frac{\frac{(n_1-1)S^2_1}{\sigma^2_1} / n_1 - 1}{\frac{(n_2-1)S^2_2}{\sigma^2_2} / n_2 - 1} \sim F_{(n_1 - 1),(n_2 - 1)}$$
poichè rapporto di due chiquadro. Risulta inoltre, semplificando:
 $$W=\frac{S_1^2}{\sigma_1^2}\frac{S_2^2}{\sigma_2^2}=\frac{\sigma_2^2}{\sigma_1^2}\frac{S_1^2}{S_2^2}$$
$w_1$ e $w_2$ saranno i quantili di ordine $\alpha / 2$ e $1 - \alpha / 2$ della distribuzione $F_{(n_1 - 1),(n_2 - 1)}$, 
esplicitamente 
$w_1 = w_{(n_1 - 1,n_2 - 1; \alpha / 2)}=\frac{1}{w_{(n_2 - 1,n_1 - 1; 1 - \alpha / 2)}}$ e 
$w_2 = w_{(n_1 - 1,n_2 - 1; 1 - \alpha / 2)}$ e quindi troviamo che $$1 - \alpha = P(w_1 \leq \frac{\sigma_2^2}{\sigma_1^2}\frac{S_1^2}{S_2^2} \leq w_2)=P\left( \frac{S_1^2 / S_2^2}{w_2} \leq \frac{\sigma_1^2}{\sigma_2^2} \leq \frac{S_1^2 / S_2^2}{w_1} \right)$$
da cui troviamo anche il relativo intervallo casuale.
Quando invece abbiamo un campione effettivamente estratto, posti $s_1^2$ e $s_2^2$ i valori assunti dalla varianza campionaria, l'intervallo di confidenza sarà
$$IC_{\frac{\sigma_1^2}{\sigma_2^2}}(1 - \alpha)=
\left[ \frac{s_1^2 / s_2^2}{w_2} ; \frac{s_1^2 / s_2^2}{w_1} \right]=
\left[ \frac{s_1^2 / s_2^2}{w_{(n_1 - 1,n_2 - 1; 1 - \alpha / 2)}} ; \frac{s_1^2 / s_2^2}{\frac{1}{w_{(n_2 - 1,n_1 - 1; 1 - \alpha / 2)}}} \right]$$ 