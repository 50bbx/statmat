%%% lezione 15 marzo %%%

\subsection{Intervalli di confidenza}
Lezione del 15/03, ultima modifica 26/03, Michele Nardin

\begin{definizione}
Sia $(X_1,...,X_n)$ un campione casuale da una distribuzione con funzione di ripartizione $F_X (x,\vartheta)$, $\vartheta \in \Theta$. 
Definiamo Statistica Pivot una funzione $Q((X_1,...,X_n),\vartheta)$ tale che
\begin{enumerate}
\item Q è funzione del campione casuale e del parametro $\vartheta$ (parametro su cui si vuol fare inferenza)
\item Q non contiene parametri incogniti oltre a $\vartheta$
\item la distribuzione di Q, $F_Q$, è completamente nota (ossia non dipende da $\vartheta$)
\item Q è invertibile rispetto a $\vartheta$
\end{enumerate}
\end{definizione}

\textbf{Esempi:}
Campione casuale da $N(\mu,\sigma^2)$:
\begin{enumerate}
\item Supponiamo di conoscere la varianza: allora un esempio di statistica pivot è $$Z_n=\frac{\bar{X}_n - \mu}{\sigma / \sqrt{n}} $$ la quale, grazie all'ipotesi di campionamento da vc normale, ha distribuzione N(0,1).
\item Supponiamo di non conoscere la varianza: in tal caso, al posto della varianza usiamo lo stimatore varianza campionaria $S_n^2$, il quale è non distorto (già dimostrato) e consistente (infatti $MSE_{\sigma^2}(S_n^2)=Var(S_n^2) + B^2(S_n^2)=\frac{2\sigma^4}{n-1} \rightarrow 0$) e quindi la statistica pivot in questione sarà $$Q=\frac{\bar{X}_n - \mu}{\S_n / \sqrt{n}}$$ la quale (dimostreremo che) ha distribuzione t-student con n-1 gradi di libertà.
\end{enumerate}

Siano $(X_1,...,X_n)$ da una vc avente cdf $F_X(x,\vartheta)$. Vogliamo stimare $\vartheta$: per farlo usiamo uno stimatore $T_n$.
Quando il campione casuale sarà effettivamente estratto, avremo una n-upla di valori reali $(x_1,...,x_n)$, i quali permettono di calcolare l'effettivo valore della stima: è impensabile però che la stima coincida esattamente con il valore (se X ha distribuzione continua $\mathbbm{P}(T_n=\vartheta)=0$!). Dobbiamo quindi associare a $T_n$ un margine di errore.

\textbf{Esempio introduttivo (exit poll)}
Vogliamo stimare la proporzione $p_i$ dei voti ricevuti dall'iesimo partito sul totale. Il nostro problema sarà quello di trovare un intervallo centrato nella stima $\hat{p_i}$ , ed un margine d'errore, $ME$, tale per cui, ad un fissata soglia di probabilità $\alpha$ si abbia $$P(p_i \in (\hat{p_i} - ME,\hat{p_i} + ME)=1 - \alpha$$
\subsubsection{Costruzione generale}
In generale, sia $\vartheta_0$ il valore vero del parametro $\vartheta$ che vogliamo stimare, e per semplicità assumiamo che $T_n$ sia un suo stimatore tale che 
$$\sqrt{n}(T_n - \vartheta_0)\stackrel{d}{\rightarrow}N(0,\sigma_{T_n}^2)$$
Per il momento assumiamo di conoscere $\sigma_{T_n}^2$, sicché
$$Z_n=\frac{\sqrt{n}(T_n - \vartheta_0)}{\sigma_{T_n}} \stackrel{a}{\sim}N(0,1)$$ Bisogna notare che $Z_n$ è una statistica pivot.
Fissato $\alpha \in (0,1)$, consideriamo i quantili della distribuzione N(0,1), $\pm z_{\alpha / 2}$ (ossia quei valori tali per cui, se $X \sim N(0,1)$, $P(-z_{\alpha / 2} \leq X \leq z_{\alpha / 2})=1-\alpha$). Possiamo affermare che, per n sufficientemente grande, (il simbolo $\stackrel{.}{=}$ indica un'uguaglianza approssimata) $$P(-z_{\alpha / 2} \leq Z_n \leq z_{\alpha / 2})\stackrel{.}{=}1-\alpha$$ da cui
$$P(-z_{\alpha / 2} \leq \frac{\sqrt{n}(T_n - \vartheta_0)}{\sigma_{T_n}} \leq z_{\alpha / 2})\stackrel{.}{=}1-\alpha $$ e ancora
$$P(T_n - z_{\alpha / 2} \frac{\sigma_{T_n}}{\sqrt{n}} \leq \vartheta_0 \leq T_n+z_{\alpha / 2} \frac{\sigma_{T_n}}{\sqrt{n}})\stackrel{.}{=}1-\alpha$$
Possiamo quindi definire un intervallo casuale, $$IC=\left[T_n - z_{\alpha / 2} \frac{\sigma_{T_n}}{\sqrt{n}},T_n + z_{\alpha / 2} \frac{\sigma_{T_n}}{\sqrt{n}}\right]$$ (è casuale perchè per $T_n$ è una vc). $IC$ è uno Stimatore Intervallare.
Si può affermare che $P(\vartheta \in IC) \stackrel {.}{=} 1 - \alpha$.

$\textbf{Nomenclatura}:$
$z_{\alpha / 2}$ si dice Fattore di Affidabilità,
$\frac{\sigma_{T_n}}{\sqrt{n}}$ si dice Standard Error dello stimatore $T_n$.

Sia ora $(x_1,...,x_n)$ una determinazione campionaria (ossia i dati effettivamente osservati da un campione casuale) (cioè una n-upla) e sia $T_n(x_1,...,x_n)=t_n$ l'effettivo valore assunto dallo stimatore.
Definiamo \textbf{l'intervallo di confidenza con probabilità di copertura $1-\alpha$} $$IC_\vartheta (1-\alpha) := \left[t_n - z_{\alpha / 2} \frac{\sigma_{T_n}}{\sqrt{n}}, t_n + z_{\alpha / 2} \frac{\sigma_{T_n}}{\sqrt{n}}\right]$$
La probabilità di copertura viene anche detta livello di confidenza.

Nella pratica, $\sigma^2_{T_n}$ non è noto a priori. Possiamo però usare lo stimatore varianza campionaria di $T_n$, $S^2_{T_n}$, il quale sappiamo che converge in probabilità a $\sigma^2_{T_n}$. Allora, per il teorema di Slutsky, troviamo che 
$$Z_n=\frac{\sqrt{n}(T_n - \vartheta_0)}{S_{T_n}} = \frac{\sqrt{n}}{S_{T_n}} T_n - \frac{\sqrt{n}}{S_{T_n}} \vartheta_0 \stackrel{d}{\rightarrow}N(0,1)$$ Possiamo quindi ripetere il ragionamento fatto poco sopra usando la varianza campionaria al posto di $S^2_{T_n}$, e quindi costruire l'intervallo di confidenza con probabilità di copertura pari a $1 - \alpha$ come $$IC_\vartheta (1-\alpha) := \left[t_n - z_{\alpha / 2} \frac{S_{T_n}}{\sqrt{n}}, t_n + z_{\alpha / 2} \frac{S_{T_n}}{\sqrt{n}}\right]$$
\subsubsection{Intervallo di confidenza per la media $\mu$}
Sia $(X_1,...,X_n)$ un campione casuale, media e varianza incognite. Siano $\bar{X}_n$ e $S^2_n$ gli stimatori di media e varianza della popolazione. Allora per il TLC e per il teorema di Slutsky si ha che 
$$\frac{\sqrt{n}(\bar{X}_n - \mu)}{S_n} \stackrel{d}{\rightarrow}N(0,1)$$
che è una statistica pivot. Quindi l'intervallo di confidenza con probabilità di copertura $1-\alpha$ (sempre approssimato) sarà
$$IC_\mu(1-\alpha)=\left[\bar{X}_n - z_{\alpha / 2} \frac{S_n}{\sqrt{n}}, \bar{X}_n + z_{\alpha / 2} \frac{S_n}{\sqrt{n}}\right]$$
\subsubsection{Intervallo di confidenza per una proporzione p}
Sia $(X_1,...,X_n)$ un campione casuale da $b(1,p)$ e sia $\hat{p}_n=\frac{1}{n} \sum_{i=1}^n X_i$ lo stimatore (corretto e consistente) di p. Troviamo che per il TLC e per la WLLN $$\frac{\sqrt{n}(\hat{p}_n - p)}{\sqrt{\hat{p}_n(1-\hat{p}_n)}} \stackrel{d}{\rightarrow}N(0,1)$$
e quindi l'intervallo di confidenza con probabilità di copertura $1-\alpha$ approssimato sarà
$$IC_p(1-\alpha)=
\left[\hat{p}_n -
 z_{\alpha / 2} 
 \sqrt{\frac{\hat{p}_n(1-\hat{p}_n)}{n}},
  \hat{p}_n + z_{\alpha / 2}\sqrt{\frac{\hat{p}_n(1-\hat{p}_n)}{n}} \frac{S_n}{\sqrt{n}}\right]$$